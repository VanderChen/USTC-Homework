%!TEX program = pdflatex
\documentclass[UTF8]{article}

\usepackage[UTF8]{ctex}
\usepackage{amsmath}
\usepackage{enumerate}
\usepackage{amssymb}
\usepackage{graphicx}
\usepackage{booktabs}
 
\title{Homework 11.23}
\author{xxx}
\date{}
\begin{document}
\maketitle
\section{Question 53}
\paragraph{Question}
Determine the conjugate of each of the following
partitions,
\begin{enumerate}[(a)]
    \item $ 12 = 5 + 4 + 2 + 1 $
    \item $ 15 = 6 + 4 + 3 + 1 + 1 $
    \item $ 20 = 6 + 6 + 4 + 4 $
    \item $ 21 = 6 + 5 + 4 + 3 + 2 + 1 $
    \item $ 29 = 8 + 6 + 6 + 4 + 3 + 2 $
\end{enumerate}
\paragraph{Answer}
    \begin{enumerate}
        \item $ 12 = 4 + 3 + 2 + 2 + 1 $
        \item $ 15 = 5 + 3 + 3 + 2 + 1 + 1 $
        \item $ 20 = 4 + 4 + 4 + 4 + 2 + 2 $
        \item $ 21 = 6 + 5 + 4 + 3 + 2 + 1 $
        \item $ 29 = 6 + 6 + 5 + 4 + 3 + 3 + 1 + 1 $
    \end{enumerate}


\section{Question 54}
\paragraph{Question}
Prove that conjugation reverses the order of majorization,
that is, if $ \lambda $ and $ \mu $ are partitions of $ n $ and $ \lambda $ is majorized by $ \mu $, then $ \mu^{*} $ is majorized
by $ \lambda^{*} $.
\paragraph{Answer}
Let us view the Ferrers diagram for $ \lambda $ and $ \mu $ as contained in a $ n \times n $ box and justied
to the North-West. Consider the $ 2n - 1 $ NW-SE diagonals in this box. One checks that the following are equivalent:
\begin{enumerate}[i]
    \item $ \lambda $ is majorized by $ \mu $
    \item for each NW-SE diagonal the number of dots in $ \mu $ that lie on or above the diagonal is
    at least the number of dots in $ \lambda $ that lie on or above the diagonal.
\end{enumerate}
\end{document}