%!TEX program = pdflatex
\documentclass[UTF8]{article}

\usepackage[UTF8]{ctex}
\usepackage{enumerate}
\usepackage{graphicx}

\title{第七、八章作业}
\author{SA18225036 陈旻}
\begin{document}
\date{}
\maketitle
\section{第七章}
\subsection{第4题}
\paragraph{答案}
$ 120*8=960 $,因此第一个字节的编号是960,但无法确定最后一个字节的编号,除非知道数据的长度。
\subsection{第6题}
\paragraph{答案}
~\\
首部最多40字节,选项格式中类型、长度各1字节。路由最多可存9个。因为标识位是1,需要记录IP地址和时间戳,所以只能存放4对地址和时间戳。
\subsection{第16题}
\paragraph{答案}
\begin{enumerate}[a.]
    \item 第一个字节编号:$ 200*8=1600 $,总长200,首部长度20,这个数据报中共有180字节。最后一个字节编号:$ 1600+180-1=1779 $。
    \item 因M位是0,所以是最后一个分片。
\end{enumerate}
\section{第八章}
\subsection{第2题}
\paragraph{答案}
~\\
长度为28个字节:  \\
硬件类型:2字节 \\
协议类型:2字节 \\
硬件地址长度:1字节 \\
协议地址长度:1字节 \\
操作类型:2字节 \\
源MAC地址:6字节 \\
源IP地址:4字节 \\
目的MAC地址:6字节 \\
故长度总和$ 2+2+1+1+2+6+4+6=28 $字节
\subsection{第5题}
\paragraph{答案}
\end{document}