%!TEX program = pdflatex
\documentclass[UTF8]{article}

\usepackage[UTF8]{ctex}
\usepackage{enumerate}

\title{第三、四章作业}
\author{SA18225036 陈旻}
\begin{document}
\date{}
\maketitle
\section{第三章}
\subsection{第五题}
\paragraph{答案}
因为最小长度为46字节,所以还应填充4字节。
\subsection{第六题}
\paragraph{答案}
不行,应总共发送2个帧,第一帧1500字节,第二帧46字节。
\subsection{第七题}
\paragraph{答案}
\begin{enumerate}[a.]
    \item 相同点 
    \begin{enumerate}[1]
        \item 每一个站点对于信道都平等
        \item 每一个站点都能侦听信道
    \end{enumerate}
    \item 不同点
    \begin{enumerate}[1]
        \item CDMA/CD :站点在侦听到线路上没有信号时即可发送 \\
              CDMA/CA :站点需要告诉其他站点,它需要占用线路一段时间
        \item CDMA/CD :碰撞有可能发生 \\
              CMDA/CA :碰撞不会发生
    \end{enumerate}
\end{enumerate}
\section{第四章}
\subsection{第一题}
\paragraph{答案}
\begin{enumerate}[a.]
    \item 优点 
    \begin{enumerate}[1]
        \item 发送简单
        \item 无资源耗尽问题
    \end{enumerate}
    \item 缺点
    \begin{enumerate}[1]
        \item 需要自己解决可靠通信问题,有可能由拥塞导致包丢失
        \item 包可能乱序到达,因为需要上层重新排序
    \end{enumerate}
\end{enumerate}
\subsection{第二题}
\paragraph{答案}
\begin{enumerate}[a.]
    \item 优点 
    \begin{enumerate}[1]
        \item 自动处理分组丢失
        \item 自动验证数据差错
        \item 处理连接状态
        \item 拥有更稳定的连接质量
    \end{enumerate}
    \item 缺点
    \begin{enumerate}[1]
        \item 对每个连接都有一个单独的套接字,耗费更多的资源,存在资源耗尽问题
        \item 在空闲的连接上不发送任何分组
    \end{enumerate}
\end{enumerate}
\subsection{第六题}
\paragraph{答案}
\begin{enumerate}
    \item 该部分的分组路径可能仍然使用无连接的服务
    \item 网络层的协议在设计时就涉及这两个地址,并且要改变这一现状可能需要一定的时间
\end{enumerate}
\end{document}