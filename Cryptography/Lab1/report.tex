%!TEX program = pdflatex
\documentclass[UTF8]{article}

\usepackage[UTF8]{ctex}
\usepackage{amsmath}
\usepackage{enumerate}
\usepackage{amssymb}
\usepackage{graphicx}
\usepackage{booktabs}
 
\title{实验报告1}
\author{陈旻 SA18225036}
\date{}
\begin{document}
\maketitle
\section{具体思路}
\paragraph{}
根据提示,空格space的ASCII码二进制形式为0010 0000。大写字母A~Z的ASCII码二进制形式为0100 0001~01011010,小写字母a~z的ASCII码二进制形式为 0110 0001~01111010。因此若用space和字母做xor操作,则相当于字母切换大小写,而两个字母做xor操作,结果将不在字母范围内。另外由于xor操作的特点,将两个密文做xor操作相当于将两个密文对应的明文做xor操作,如果结果中某个位置出现字母,则说明这两个明文的其中一个在该位置可能为空格,所以基本思路是对11个密文分别做两两xor操作,然后通过结果判断不同明文中可能存在空格的位置,然后将对应位置上的密文和space做xor操作,就可得到对应位置的密钥信息,当获取足够多的密钥信息后,就可以对最后一个目标密文进行解密了。

需要注意的是两个密文xor的结果中出现的字母不一定是space字符和字母xor的结果,从字母的ASCII码可知,小写字母的前四bit是0110,大写字母的前四bit是0100,所以如果某个字符的ASCII码前四bit是0010或0011这样的形式,那么它和字母xor的结果也可能是另外一个字母,而文本中经常出现的“?”,“!”和“:”等字符就符合这样的特征。另外如果两个文本正好在同一个位置上有空格,则xor操作后该位置得到的字符将是“NULL”。所以对于特定密文,要判断其对应明文可能存在空格的位置的话,我的想法是将该密文和其他10个密文分别做xor操作,每一次操作后记录结果中出现字母的位置,得到一个集合,最终一共得到10个集合,某一个位置在这10个集合中出现的次数越多,越说明该密文对应的明文在该位置可能是空格,因此空格位置的判定条件就是该位置在10个集合中出现的次数。
\end{document}