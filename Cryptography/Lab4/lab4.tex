%!TEX program = pdflatex
\documentclass[UTF8]{article}

\usepackage[UTF8]{ctex}
\usepackage{amsmath}
\usepackage{enumerate}
\usepackage{amssymb}
\usepackage{graphicx}
\usepackage{booktabs}
 
\title{实验报告4}
\author{陈旻 SA18225036}
\date{}
\begin{document}
\maketitle
\section{具体思路}
\paragraph{}
分解第一个和第二个数时安装题目里的公式做就行
$$ A = \lfloor \sqrt{N} \rfloor $$

由于$ A $是$ p $和$ q $的中点,所以存在一个$ x $使得$ p = A - x $以及$ q = A + x $。
又因为$ N = pq = (A - x)(A + x) = A^2 - x^2 $,因此$ x = \sqrt{A^2 - N} $。
现在,根据$ x $和$ A $,你可以找到$ N $的$ p $和$ q $。于是已经分解出了$ N $。

要注意的是gmpy2模块的精度要取得足够大,否则sqrt()函数的参数为mpfr型时,即使该参数的平方根也是整数,得出的结果可能是很趋近那个平方根的浮点数,导致最终结果错误,所以计算x值时要注意确认设定的精度可以得到正确的x。程序中精度取的是1200(gmpy2.get\_context().precision = 1200)。

分解第三个数时,要注意$ A=(3p+2q)/2 $,即A是3p和2q的中点,并且3p是奇数,2q是偶数,所以A一定不是整数,并且因为除以2,A的小数部分是0。通过公式的推导,可以知道 A和$ \sqrt{6N} $差距小于$ 1/(8*\sqrt{6})$,所以A的值等于 $\lfloor \sqrt{6N} \rfloor-0.5$,取得A值后,计算$ x=\sqrt{a^2-6N}$,计算出x后,A-x可能为3p或2q,判断一下A-x是否被3除尽就可确定。

最后一个问题是RSA解密,通过对N的分解因式可以计算 ,然后可以计算出私钥,解密的结果是很长一串数,将其转换成16进制编码的字符串,decode('hex')以后就是PKCS1编码后的明文,分隔符之后的就是原始信息。

\end{document}